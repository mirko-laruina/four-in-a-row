\section{Communication Protocol}
\subsection{Description}
\subsubsection{Handshake Protocol}
\subsubsection{Encrypted Message Protocol}

\subsection{BAN-logic proof}
\subsubsection{Real protocol}
\begin{align*}
    &M1 \; A \rightarrow B : A, B, N_A, K_A^E \\
    &M2 \; B \rightarrow A : B, A, N_B, K_B^E, \{h(A,B,N_A,N_B,K_A^E,K_B^E)\}_{k_B^{-1}} \\
    &M3 \; A \rightarrow B : \{h(A,B,N_A,N_B,K_A^E,K_B^E)\}_{k_A^{-1}}
\end{align*}

\subsubsection{Assumptions}
\begin{align*}
    \begin{split}
        &A \believes \pubkey{K_B} B \\
        &A \believes \fresh{\pubkey{K_A^E}A} \\
        &A \believes \fresh{N_A} \\
        &A \believes B \controls{\pubkey{K_B^E}B} \\
        &A \believes B \controls{N_B}
    \end{split}
    \begin{split}
        &B \believes \pubkey{K_A} A \\
        &B \believes \fresh{\pubkey{K_B^E}B} \\
        &B \believes \fresh{N_B} \\
        &B \believes A \controls{\pubkey{K_A^E}A} \\
        &B \believes A \controls{N_A}
    \end{split}
\end{align*}
\subsubsection{Expected Conclusions}
\begin{align*}
    \begin{split}
        &A \believes (N_A,N_B,\pubkey{K_A^E}A,\pubkey{K_B^E}B) \\
        &A \believes B \believes (N_A,N_B,\pubkey{K_A^E}A,\pubkey{K_B^E}B)
    \end{split}
    \begin{split}
        &B \believes (N_A,N_B,\pubkey{K_A^E}A,\pubkey{K_B^E}B) \\
        &B \believes B \believes (N_A,N_B,\pubkey{K_A^E}A,\pubkey{K_B^E}B)
    \end{split}
\end{align*}

\subsubsection{Idealized protocol}
\begin{align*}
    &M1 \; A \rightarrow B : N_A, \pubkey{K_A^E}A \\
    &M2 \; B \rightarrow A : \{N_A,N_B,\pubkey{K_A^E}A,\pubkey{K_B^E}B\}_{k_B^{-1}} \\
    &M3 \; A \rightarrow B : \{N_A,N_B,\pubkey{K_A^E}A,\pubkey{K_B^E}B\}_{k_A^{-1}}
\end{align*}

\subsubsection{Proof}
\paragraph{After message 1} 
From the first message, B just knows SOMEONE sent him a nonce and an ephemeral public key. 
He will use these values in later exchanges but at the moment he knows nothing 
about the sender.
\begin{align*}
    M1 \; &A \rightarrow B : N_A, \pubkey{K_A^E}A & \text{B does not know anything about the sender} \\
    &B \sees (N_A,\pubkey{K_A^E}A)
\end{align*}

\paragraph{After message 2} 
From this message A is able to verify B's identity and has confirmation that B
actually knows the nonce and her ephemeral public key. A also receives B's 
nonce and ephemeral public key.
\begin{align*}
    M2 \; &B \rightarrow A : \{N_A,N_B,\pubkey{K_A^E}A,\pubkey{K_B^E}B\}_{k_B^{-1}} & \text{apply message meaning postulate} \\
    &A \believes B \oncesaid (N_A,N_B,\pubkey{K_A^E}A,\pubkey{K_B^E}B) & \text{apply nonce verification postulate}  \\
    &A \believes B \believes (N_A,N_B,\pubkey{K_A^E}A,\pubkey{K_B^E}B) & \text{furthermore, due to juristiction postulate} \\
    &A \believes (N_B,\pubkey{K_B^E}B)
\end{align*}

\paragraph{After message 3} 
From this message B is able to verify A's identity and has confirmation that A
actually knows the nonce and his ephemeral public key.
\begin{align*}
    M3 \; &A \rightarrow B : \{N_A,N_B,\pubkey{K_A^E}A,\pubkey{K_B^E}B\}_{k_A^{-1}} & \text{apply message meaning postulate} \\
    &B \believes A \oncesaid (N_A,N_B,\pubkey{K_A^E}A,\pubkey{K_B^E}B) & \text{apply nonce verification postulate}  \\
    &B \believes A \believes (N_A,N_B,\pubkey{K_A^E}A,\pubkey{K_B^E}B) & \text{furthermore, due to juristiction postulate} \\
    &B \believes (N_A,\pubkey{K_A^E}A)
\end{align*}

\subsubsection{Final remarks}
It is worthwhile to note that the nonces $N_A,N_B$ play no role in the proof of
the protocol. 
In fact, the nonce verification postulate could be applied also on the ephemeral 
public keys (since they are fresh too). 
Furthermore, the shared DH secret is guaranteed to always be different since 
both parties choose an ephemeral public key at random at every handshake.
Therefore, we could remove the random nonces from the protocol without loss of 
security. 
