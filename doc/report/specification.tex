\section{Project Specification}
The project consists in the implementation of a four-in-a-row application. 
The user is proposed with the possibility of playing offline (against the computer) or online.

When playing online, the communication is encrypted and authenticated.
Both the server and the clients have a certificate. The server holds all the certificates of the users registered to it and, when a game has to start between two players, sends the certificate of the other party to the users.

The two users establish a connection between themselves and can start playing their game.

We can distinguish two phases during our application lifetime:
\begin{itemize}
	\item Server lobby
	\item Peer to peer
\end{itemize}

During the first phase, the user can see all the active users and can decide who (s)he wants to challenge.
The peer to peer phase is the one in which the match happens. After it is concluded, both players are brought back to the server lobby.

Notice that we have chosen to implement the peer to peer phase as a standalone gaming mode, even if it wasn't a functional requirement, since it is useful for debugging purposes and its implementation is quite easy. Between clients no certificate exchange is possible so the users can challenge only peers of which they know the certificate.

\subsection{Application run}
When a user runs the application, (s)he will be proposed with a welcoming screen in which all the commands are listed.
\begin{itemize}
	\item \textit{server host port [certificate]}: the user connects to the server listening on the provided host and port. The certificate is an optional parameter. If it is specified, it will be the one used for the communication, otherwise a \texttt{CERTIFICATE\_REQUEST} message will be issued to which the server will reply with its own certificate. The user is then brought in server lobby.
	\item \textit{peer host port certificate}: the user connects to the peer listening on the provided host and port. The certificate has to be provided, otherwise the communication will not be possible. If the connection is successful, the match can begin.
	\item \textit{peer listen\_port certificate}: the user will listen on the specified port and will wait for a connection from the peer having the specified certificate. The match will begin when the other peer connects to the host.
\end{itemize}

\subsection{Server lobby}
In this mode, the user connects to a server. The server is responsible of handling all the active users, forwarding incoming challenges and setting up a match if the parties agree on it.

An active user is a player actively connected to the server, he can be free or playing a match against another user.

In particular, the server accepts different commands:
\begin{itemize}
	\item list : shows the list of all the active users
	\item challenge username:  challenges the user with the specified username, if (s)he is available (s)he will be asked to accept or decline the challenge.
	\item exit: ends the communication with the server
\end{itemize}

When a challenge is accepted, the user receives the certificate of the challenger (so does the other player) and he can start communicating with him as if it was a peer to peer connection.

\subsection{Peer to peer}
The peer to peer phase, which can be standalone as previously specified or following the starting of a match decided in the server lobby, is the phase in which the users connect directly to eachother.

In order to be possible, one of the peers should listen to a port and the other should connect to it. In server mode, the information on the host and port are forwarded by the server itself, while if connecting directly to the peer, we have to have this knowledge beforehand.

The existence of NATs makes this mode difficult to use when not in the LAN.

\subsection{Remarks}
Since there are no main differences between the server and clients capabilities, the protocol used is the same both for the communication server-client than peer-peer.




