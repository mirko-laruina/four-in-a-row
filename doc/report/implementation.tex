\section{Implementation}

\subsection{Development}
During all the development of the application, we posed particular attention to the modularity of the software.

In particular, our development was divided in different phases which acted as building blocks:
\begin{itemize}
	\item \textit{offline} : the basic game was implemented (CLI, win conditions etc.)
	\item \textit{networking} : a new layer was added and allowed playing over the network in an ``insecure'' way (all messages in the clear). To avoid having to modify the code heavily in the next phase, an interface to utilize the socket was defined (\emph{SocketWrapper}).
	\item \textit{security}: the cryptographic algorithms were implemented using the OpenSSL library and a \emph{SecureSocketWrapper} was defined and used in the client and server applications.
\end{itemize}

To speed up the development, a \emph{Makefile} and some automated tests were written to automatically compile the code and check that the behavior of the application was as designed. By simply running \texttt{make test} all three 
possible game modes are tested by spawning dummy processes.

\subsection{Source code overview}
\label{ref:modules}

The source code is organized into 5 modules:
\begin{itemize}
    \item \emph{network}: contains common utilities for managing the network. 
        The two most important abstractions are the \emph{Message}, which 
        is an abstract interface that all message classes must implement, and 
        the \emph{SocketWrapper}, which wraps an \emph{inet socket} and 
        provide functions for easily sending and receiving \emph{messages}.
    \item \emph{security}: contains the OpenSSL implementation of the 
        cryptographic functions and the \emph{SecureSocketWrapper} class,
        providing the same interface as the \emph{SocketWrapper} but using 
        encrypted messages.
    \item \emph{utils}: miscellaneous classes providing easy-to-use utility 
        functions for, e.g., securely reading from \emph{stdin} (preventing 
        buffer overflows), reading/writing to a bounded buffer (once again 
        to prevent possible buffer overflows), dumping a buffer to stdout 
        for debug purposes. It also provides an implementation of a thread-safe
        message-passing queue which is used by the server to distribute tasks
        among worker threads.
    \item \emph{server}: implementation of the server executable, along with
        some private utility classes: \emph{User}, which represent a user as 
        a finite state machine, and \emph{UserList}, which provides a convenient
        thread-safe collection of users which can be retrieved (through an
        hash map) by either their username or the socket file descriptor they 
        are bound to.
    \item \emph{client}: implementation of the client executable, along with 
        some private utility classes: \emph{Connect4}, which provides the game 
        implementation, \emph{Server}, which proxies the client communication 
        to the server by providing a more simple interface without having the 
        client implement the networking logic. 
        Furthermore, the \emph{multi\_player} and \emph{single\_player} modules 
        provide the main functions to handle the game in the different 
        circumstances.
\end{itemize}

In the following, the main workflow of the client and server is first
illustrated and then a brief explanation of some of the main utility components 
is provided. For a more in-depth description, please refer to the auto-generated
Doxygen documentation or the source code itself.

\subsection{Server}
The server is implemented as a main thread, accepting communications from 
clients and reading messages from sockets, and multiple worker threads, that 
handle the messages (including decryption). We will first describe the main 
thread and then the worker threads.

\subsubsection{Main thread}
At start-up, the main thread loads its own certificate and key, the CA's 
certificate and all client certificates. If this operation succeeds, the 
server creates a new listening socket on the given port, spawns the worker 
threads and waits for new connections using the \emph{select} system call.
On a new connection request, the server accepts the connection and adds the 
socket to the list of sockets that checked by the \emph{select}. When a client
sends data to the server, the main thread is woken and the corresponding 
message will be en-queued in the \emph{MessageQueue}, waiting for a worker
thread to elaborate it.

\subsubsection{Worker thread}
The worker thread waits on a \emph{condition} variable waiting for new messages
to be available in the \emph{MessageQueue}. Once a new message is available,
the thread starts elaborating it. First of all, it retrieves the \emph{User}
instance\footnote{at this time the user may not be logged in yet or he could 
not even have completed the handshake. In these cases, the \emph{User} instance
just represents the socket and not yet the user.} that message refers to and 
locks it with a \emph{mutex} to prevent other threads to change the \emph{User} 
state concurrently. Depending on the state of the user, messages are handled
accordingly:

\begin{itemize}
    \item \emph{Just connected}: this is the starting state of a \emph{User}.
        In this state the following messages are handled:
        \begin{itemize}
            \item \emph{Certificate request}: the server certificate is 
                forwarded to the \emph{User} by means of the \emph{certificate}
                message.
            \item \emph{Client hello}: the client has begun an \emph{handshake}
                with the server. The server replies with a \emph{server hello}
                message.
            \item \emph{Client verify}: the client is terminating the 
                \emph{handshake}. If verification succeeds, the user is 
                considered authenticated and is moved to the \emph{Securely 
                connected} state.
        \end{itemize}
    \item \emph{Securely connected}: this is the first state a user is brought 
        upon authentication. In this state only one message is valid:
        \begin{itemize}
            \item \emph{Register}: the \emph{User} communicates the server that
                she is ready for receiving challenges and is moved to the 
                \emph{registered} state without further action. Note that this
                message may not be necessary, however it has been kept for 
                compatibility with the non-secure version of the game.
        \end{itemize}
    \item \emph{Available}: in this state the \emph{user} is fully
        operational: he may be returned in the user list to other users, he may
        be challenged by other users and challenge other users. In this state 
        the following messages are valid:
        \begin{itemize}
            \item \emph{User list request}: the \emph{User} asks the server
                to tell him which users are online. The server replies with the 
                \emph{user list}.
            \item \emph{Challenge}: the \emph{User} tells the server he wants 
                to challenge an \emph{opponent}. The server moves both 
                \emph{Users} to the \emph{Challenged} state and forwards the 
                challenge to the \emph{opponent} with a \emph{challenge forward}
                message.
        \end{itemize}
    \item \emph{Challenged}: in this state the \emph{user} cannot receive any 
        further challenge and is either waiting for a reply from the 
        \emph{opponent} or deciding whether to accept it. In this state only 
        the following message is handled.
        \begin{itemize}
            \item \emph{Challenge reply}: the \emph{User} tells the server that
                she accepts or refuses the challenge. In the first case, 
                the user also indicates on which port he will listen for a 
                connection from the opponent. If the challenge is accepted, 
                both users are sent a \emph{game start} message indicating 
                the certificate of the \emph{opponent} and, in the case of 
                the user that sent the challenge in the first place, also 
                the IP address and port which the opponent is listening for 
                a connection. Finally, both users are moved to the 
                \emph{playing} state. In the case the challenge is refused, 
                both users are sent a \emph{challenge cancel} message and 
                are moved back to the \emph{available} state.
        \end{itemize}
    \item \emph{Playing}: in this state the \emph{user} cannot receive any 
        challenge since it's playing with another user. In this state only 
        one message can be received:
        \begin{itemize}
            \item \emph{Game end}: the user signals the server that he is now 
                available and is moved to the \emph{available} state.
        \end{itemize}
    \item \emph{Disconnected}: the user connection dropped. The \emph{User}
        instance is waiting to be destroyed and all pending messages from the 
        users are ignored.
\end{itemize}

Furthermore, note that:
\begin{itemize}
    \item In case an unexpected message is received, the message is simply 
        ignored. 
    \item In case a challenge is not possible, either because the user or the   
        opponent is in a state where he cannot receive challenges or because 
        the \emph{user}/\emph{opponent} disconnected in the meanwhile, 
        a \emph{game cancel} message is sent to both users (or just one).
\end{itemize}

Once the message handling is completed, the message is deleted, the \emph{User} 
\emph{mutex} unlocked and the \emph{MessageQueue} is polled again. 
However, if the message handling completed with problems that indicate that 
the user is no longer connected, the \emph{User} is also moved to the 
\emph{disconnected} state before releasing the lock. It is important to note 
that the worker thread must not delete the \emph{User} instance because another
thread may hold a reference to it. For this reason, the \emph{UserList} keeps
track of outstanding references to every user and is in charge of deleting 
disconnected \emph{Users} that are no longer referenced.

\subsection{Client}
At start-up, the client loads his certificate and key (inserting a password if 
necessary) and then waits for the user to decide a playing mode: server, 
peer (waiting for a connection), peer (connecting to the other peer) and 
offline (used for debug purposes, the opponent has no logic).

\subsubsection{Server mode}
Once the user selects this mode, he also indicates the address and port of the 
server. The client then sends a \emph{certificate request} to the server and 
waits for the corresponding \emph{certificate} reply. It then initiates the 
\emph{handshake} with the server, sending the \emph{client hello} message. Once 
the corresponding \emph{server hello} message is received, the signature is 
checked against the certificate and it sends back the \emph{client verify} 
message to complete the authentication. Now that the \emph{handshake} is 
completed, the client sends the \emph{register} message and waits for either 
user input from stdin or server messages from the network using the 
\emph{select} system call. Let's first describe the possible user commands that 
can be given from stdin:
\begin{itemize}
    \item \emph{list}: the client sends a \emph{user list request} message,
        waits for the corresponding \emph{user list reply} message from 
        the server containing the list of users and then displays the list 
        to the user.
    \item \emph{challenge}: the client sends a \emph{challenge} message 
        to the server indicating that he wants to challenge an \emph{opponent}
        and waits for either a \emph{game start} or a \emph{game cancel} 
        message. In the first case, the client enters the \emph{connect to 
        peer} peer-to-peer mode. In the second case, the client restarts polling
        with the \emph{select} system call.
\end{itemize}

In addition, the following messages may be received from the server:
\begin{itemize}
    \item \emph{challenge forward}: the server is forwarding a challenge from 
        another user that the user can either accept or refuse. If the 
        challenge is accepted, the client sends the server the port he will 
        wait for the peer connection in the \emph{challenge response} 
        message and enters the \emph{wait for peer} peer-to-peer mode.
        Otherwise, it will just indicate that the challenge is to be
        refused.
\end{itemize}

Note that when the client leaves the \emph{server mode} to enter one of the 
two \emph{peer-to-peer modes} the connection with the server is kept alive 
and will be reused when the game finishes to send the \emph{game end} message.
The client is then brought back to the \emph{server mode} where the user 
can list users, challenge other users and receive challenges.

This mode is implemented in the \emph{client/server\_lobby.cpp} file.

\subsubsection{Peer-to-Peer mode}
The peer-to-peer mode is divided into two roles: server, which waits for a peer
connection, and client, which connects to a peer. Once a secure connection is 
established between the peers, the client enters the common \emph{playing} mode.

This mode is implemented in the \emph{client/multi\_player.cpp} file.

\paragraph{Server: wait for peer}
Once entering this mode, the client opens a new listening socket waiting for 
a connection from the peer. Once the peer connects, the \emph{handshake} is 
performed as described before and the client enters the \emph{playing} mode as 
the first player to place a move.

\paragraph{Client: connect to peer}
Once entering this mode, the client tries to connect to the peer (it will retry 
every second for up to 10 trials). 
Once the peer connects, the \emph{handshake} is performed as described before 
and the client sends a \emph{start game peer} message to start the game.
The client then enters the \emph{playing} mode as the second player to place a 
move.

\paragraph{Playing}
For each turn until the end of the game, the client performs the following 
operations depending on whose turn it is:
\begin{itemize}
    \item if it is its turn, it will ask the user to choose a column. If the 
        column is valid, the choice is sent to the peer with a \emph{move}
        message. Once the message is sent, the move is placed on the board
        and the turn is switched.
    \item if it is the turn of the opponent, it will wait for a \emph{move}
        message. Once the message is received, the move is placed on the board 
        and the turn is switched.
\end{itemize}
 
\subsection{Main common utility component}

\paragraph{Args} 
The \emph{Args} class is used to safely parse user input from the standard 
input. By securely, we mean that the user input is read in a \emph{C++ string}
thus preventing possible buffer overflows. The class then returns an 
``argc/argv''-like format.

\paragraph{buffer\_io}
The \emph{buffer\_io} module provides functions to read/write from a bounded 
buffer. Before reading/writing a possible buffer overflow is checked and an 
error return value is returned.

\paragraph{crypto and crypto\_utils} 
The \emph{crypto} module provides the implementation of the cryptographic 
functions using OpenSSL, while the latter provides some utility functions 
related to OpenSSL objects (e.g. serializing a public key or a certificate).

\paragraph{messages}
The \emph{messages} module provides the implementation of all messages. All 
messages are a class that implements the \emph{Message} interface which 
provides a common method to read or write from/to a buffer.
By exploiting polymorphism, it is then easy to read a message from a buffer and
to cast it to the correct class, through the \emph{readMessage} function.

\paragraph{SocketWrapper}
The \emph{SocketWrapper} provides convenient methods for sending and receiving
\emph{messages}. Every message is encoded in the same way: 2 bytes for the 
total length of the message, one byte for the message type (used to select 
the class to use to read from it) and the body of the message. 
When sending a message, the message is written to the output buffer using the
\emph{write} method of the message, then the length is added in the heading 
and the message is sent through the \emph{send} system call. 
When reading a message, first the first two bytes are read to detect the 
message length using the \emph{receive} system call with the \emph{WAIT\_ALL}
flag, then the remainder of the message is read to the input buffer using the
same system call. The buffer is then passed to the \emph{readMessage} function
that will return a pointer to \emph{Message}, which has been cast to the 
correct type. 

\paragraph{SecureSocketWrapper}
The \emph{SocketWrapper} provides the same abstraction of \emph{SocketWrapper}
but with the additional layer of security. Only a selected range of messages
is allowed to be sent or received in plain text (i.e. \emph{certificate 
request}, \emph{certificate}, \emph{client hello}, \emph{server hello},
\emph{client verify}). All other messages require peer authentication through 
the handshake protocol, which is managed by the \emph{SecureSocketWrapper}
itself which also manages all secure connection state variables (keys, IVs, 
sequence numbers).
To send any other message, encryption is required. In order to perform 
encryption, the message is first written to a
plain-text buffer, as is done in the \emph{SocketWrapper} sending method, 
then the message is encrypted with AES-256 in GCM mode using the total message
length and the message type as AAD. The resulting cipher-text and verification
tag are then used to build a \emph{secure message} which is finally sent through
the socket using a \emph{SocketWrapper}. Once a message is successfully sent, 
the write sequence number is increased.
To receive messages, the converse process is done: the \emph{secure message}
is received from the \emph{SocketWrapper}, cipher-text and tag are extracted,
cipher-text is decrypted in a decryption buffer which is passed to the 
\emph{readMessage} function that will return a pointer to \emph{Message}, which 
has been cast to the correct type. Tag is verified in the decryption function
of AES-256-GCM and, in case of verification failure, an error value is returned.
Once a message is successfully read, the write sequence number is increased.
